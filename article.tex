% !TeX spellcheck = en_US
\documentclass[]{TAACpaper}
\usepackage[linesnumbered,boxed]{algorithm2e}
\begin{document}
each code-block optionally)

%This is the comment. Further under the text features of use of style of theses are described

%We are specifying language of the paper
\selectlanguage{english}
\def\dd#1#2{\frac{\partial#1}{\partial#2}}
\section{
%%%%%%%%%%%%%%% You have to specify name of your paper here
``Investigation of tuning parameters and modification of tabu search algorithm for solving the Static Routing Courier Delivery Problem.'' 
}

\authors{
R.~Shafeyev , L.~Lyubchik 
}

\abstract{
%%%%%%%%%%%%%%% You have to insert abstract of your paper here


In the paper ....
}

%%%%%%%%%%%%%%% To define new section use a command \subsection
% We are defining "Introduction" here
\subsection{Introduction}
%%%%% Paragraph text. To separate paragraphs use empty lines.
 The VRP \cite{VRPTW} ... 

\subsection{Problem Statement}
Let $C,dim(C)=n$ is a set of vehicles and $Q,dim(Q)=m$ is a set of requests received from customers at the current time.

Suppose that the following information is known about vehicles from set $C$: \\
$\vec{P}_c$ -- vehicle position, $c \in C$;\\
$L_c$ -- vehicle capacity, $c \in C$.

Let $S$ is a set of senders, ($dim(S) = dim(Q) = m$), $R$ is a set of receivers, $dim(R) = dim(Q) = m$. Then each client request $q \in Q$ includes the following information:\\
$s_q$ -- a sender of client shipment, $s \in S$; \\
$r_q$ -- a receiver of client shipment, $r \in R$; \\
$\vec{P}_{s_q}$ -- sender position;\\
$\vec{P}_{r_q}$ -- receiver position; \\
$w_q$ -- client shipment that is required to deliver from the sender to the recipient; \\
$[t_{s}^{q}, t_{s}^{q}+ \Delta{t_{s}^{q}}]$ -- the time window within which worker must to pick up the goods from the sender;\\ 
$[t_{r}^{q}, t_{r}^{q}+ \Delta{t_{r}^{q}}]$ -- the time window within which worker must deliver the goods to the receiver.
 
Thus, each application can be represented as a tuple:
\begin{equation}
\forall q \in Q: \exists q = (s_q,r_q, \vec{P}_{s_q}, \vec{P}_{r_q}, w_q, t_{s}^{q}, \Delta{t_{s}^{q}}, t_{r}^{q}, \Delta{t_{r}^{q}})
\end{equation}
To estimate the cost of transportation between destinations defined cost function $\Omega$:
\begin{equation}
\forall i,j \in S \cup R: \exists \Omega_{i,j} = \Omega(\vec{P}_i,\vec{P}_j)
\end{equation}

Necessary to construct the optimal routes of vehicles movement for the transportation of goods from the sender to the receiver for all client requests.

\subsection{Discrete model}
Routing problem can be represented as a directed graph $G=G(V,E)$. The set $V=C\cup{S}\cup{R}$ -- nodes of the graph $G$, consisting of elements of vehicles, senders and receivers. $E$ -- dynamic set of arcs of the graph $G$, such that:
\begin{equation}
\forall e(X) \in E: e(X) = (v_i,v_j), \exists v_i \in V, v_j \in V/C
\end{equation}

Let $\{X^k\}^n_{k=1}$ is a sequence of variable matrices for each vehicle $k \in C$. Elements of the matrices take the following values:
\begin{equation}
  x^{(k)}_{i,j} = 
    \begin{cases}
	  1,&\text{the vehicle $k \in C$ moves from $i$ node to the $j$}\\
	  0,&\text{otherwise.}
    \end{cases}
\end{equation}
where: $i\in{V/C \cup {k}}, j \in V/C$.

Let us introduce the vector  of variables $ \vec{Y}  ^ k(X) $ for each vehicle $ k \ in C $. Vector elements have the following values:
\begin{equation}
\vec{y}^{(k)}_{j}(X) = 
\begin{cases}
1,&\text{the request $j \in Q$ is processed by vehicle $k \in C$}\\
0,&\text{otherwise.}
\end{cases}
\end{equation}
where: $i\in{V/C \cup {k}}, j \in V/C$.


Let $t^k_j(X^{(k)})$ -- arrival time of the vehicle $k \in C$ at the destination $j \in S \cup R$. 

The objective function takes the following form:
\begin{equation} \label{main_objective}
  F(X) = 
    \sum_{k \in C}
     \sum_{i,j\cup{V}} 
     \Omega_{ij} \cdot x_{ij}^{(k)} 
     \to min
\end{equation}

We define constraints on the objective function (\ref{main_objective}), which provided the continuity of routes:
\begin{align} 
& \sum_{k \in C}\sum_{j \in S \cup R}x^{(k)}_{i,j} \leq 1, 
\forall i \in V \label{main_cond_1}\\
& \sum_{k \in C}\sum_{i \in S \cup R \cup \{k\} } x^{(k)}_{i,j} = 1, 
\forall j \in S \cup R \label{main_cond_2}\\
& \sum_{i \in S \cup R \cup \{k\} } x^{(k)}_{i,\omega} - 
\sum_{j \in S \cup R} x^{(k)}_{\omega,j} \leq 1, 
\forall \omega \in S \cup R,  \forall k \in C \label{main_cond_3}\\
&  \sum_{i \in S \cup R / Z}\sum_{j \in Z } x^{(k)}_{i,j} > 0, 
Z=\{z \in Z: \sum_{j \in S \cup R}x^{(k)}_{j,z}>0 \}  ,\forall k \in C \label{main_cond_4}
\end{align}

The restriction (\ref{main_cond_1}) prohibits to a node in the graph $G$ has more than one output arc. The restriction (\ref{main_cond_2}) prohibits to a node has more than one input arc.  The constraint (\ref{main_cond_3}) indicates that the number of input arcs to the node can not be less than the output arcs (this constraint considers the fact that the vehicle can leave the destination only if it has visited this node). The restriction (\ref{main_cond_4}) excludes local loops.

Then next constraints synchronize values of variables $X$ and $\vec{y}$ for each request $q \in Q$ and prohibits the service of receiver before a visit to the sender:
\begin{align} 
& x^{(k)}_{s_q} + x^{(k)}_{r_q} = 2 \cdot y^{(k)}_{q}(X), \forall k \in C, q \in Q  \\
& y^{(k)}_{q}(X) \cdot (\tilde{t}^k_{r_q}(X^{(k)})-\tilde{t}^k_{s_q}(X^{(k)})\ge{0}, \forall k \in C, q \in Q
\end{align}

Define restrictions for accounting of vehicle capacity and time windows:
\begin{align} 
& \sum_{j\in{Q}} \omega_j \cdot y_{j}^{k} \leq L_k, \forall{k}\in{C}\\
& t_{s}^{q} \leq \tilde{t}^k_{s_q}(X^{(k)} \leq t_{s}^{q}+ \Delta{t_{s}^{q}}, \forall q \in Q, \label{tws_cond} \\
& t_{r}^{q} \leq \tilde{t}^k_{r_q}(X^{(k)} \leq t_{r}^{q}+ \Delta{t_{r}^{q}}, \forall q \in Q, \label{twr_cond}
\end{align}



\subsection{Description of the Algorithm}

	
\begin{algorithm}[H]
	\textbf{function TabuSearch}($x^0,l,p,N_{stop}$) \\
	// Initialize variables:	\\
	$x^{opt} = x^0$;
	$F^{opt} = F^0$;
	$\phi^{0} =  \emptyset$;
	$i=0;s=0$; \\
	\While{the breakpoint is not triggered }{
	   $N_l = N_l(x^i,\phi^i,p) $; \\
		\eIf{$N_l \ne   \emptyset $}{
			$ x^{i+1} = x^i; i = i + 1;$ \\
			\textbf{continue;}
		}
		{
			// find optimum into the neighbord $N_l$: \\
			$ x^{i+1} : F(x^{i+1}) = min\{ F(y): y \in N_l \}$;
		} 

		\If{$F(x^{i+1}) < F_{opt})$}{
			$ F_{opt} = F(x^{i+1})$;
			$ x^{opt} = x^{i+1}$;
		}
			
		
		$\phi^{i+1} = update(\phi^i)$;\\
		i = i + 1;\\
	   
	}
	\textbf{return} $x^{opt}$;
	
\caption{Pseudo-code for probabilistic tabu-search algorithm.}
\label{alg:generalGP}
\end{algorithm}

\subsection{Initialization}

\subsection{A Method of Forming Solutions Neighborhood}


\subsection{Modification}
The first generalized diagram is as follows. First, the set of solutions is constructed by Solomon. Then every solution is improved by tabu search and choose the best solution according to the objective function.
Modification of the scheme is based on the hypothesis of "On a large valley." According to this hypothesis, the average local optima are located much closer to the global than a randomly chosen point. There is a certain concentration of local optima in a small part of the feasible region, which is figuratively called a large valley. If this assumption is true, then it is advisable to remember the best solutions and based on them design new original decision. We use this idea to solve the  Routing Courier Delivery Problem.

From these solutions generated population. To do this, first sort by the value of the objective function in ascending order. Each solution gets into the population with a given probability, the probability of selection decreases with increasing atomic number. It then crosses the decisions by the following rule: the first solution is randomly selected route, then the other solutions chosen a route that does not include the customer first route, etc. Other clients added by Solomon in the selected route. Then the process is repeated.

\subsection{Conclusion}
Text...


\begin{thebibliography}{99}
%In brackets we writes the name which is used for referencing.

\bibitem{VRPTW} T.~Babb, Pickup and Delivery Problem with Time Windows // Coordinated Transportation Systems: The State of the Art. Department of Computer Science University of Central Florida Orlando, Florida, 2005, 38 p.

\bibitem{Tabu_search} O.~Braysy, M.~Gendreau, Vehicle Routing Problem with Time Windows, Part I: Route Constuction and local algorithms // Transportation science Vol.39 No. 1, 2005, p. 104-118.

\bibitem{Assignment_Problems} E.~Rainer, Assignment problems, 2009, 402 p. 

\bibitem{Goldberg_Kennedy} V.~Goldberg, R.~Kennedy, An Efficient cost scaling algotirhm for the assignment problem, Math. Program., 1995, p. 153--177.  

\bibitem{problems_Christofides}  N. Christofides, S. Eilon, An algorithm for the vehicle dispatching problem //Operational Research Quarterly, 20, 1969, p. 309–318.
\bibitem{problems_Golden}  B. Golden, E. Wasil, J. Kelly, I-M. Chao. The impact of metaheuristics on solving the vehicle routing problem: Algorithms, problem sets, and computational results. In T. Crainic and G. Laporte, editors // Fleet Management and Logistics, Kluwer, Boston, 1998 p. 33–56.
\bibitem{problems_Taillard}  E. Taillard. VRP benchmarks.\\ http://mistic.heig-vd.ch/taillard/problemes.dir/vrp.dir/vrp.html, 1993.
\end{thebibliography}



\end{document} 